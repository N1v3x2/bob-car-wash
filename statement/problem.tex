\problemname{Bob's Car Wash}

Bob owns a car wash service. Each day, people drive their cars in and get their cars hand-washed by Bob. Because Bob does such a good job, people have begun to spread word around town about his car wash service. As a result, Bob's business has become very popular. It's reached the point that he cannot possibly wash everyone's car on the same day that they arrive. 

Bob wants to come up with a new system for washing cars. While most businesses do first-come-first-serve, Bob thinks that's too inefficient. He only has a limited amount of time each day, and he wants to maximize his profit each day from washing cars. He knows that some car models take longer to wash (e.g. big cars or fancy sports cars), but their owners may also be willing to pay more to compensate. 

On the other hand, each customer wants to get their car washed as quickly as possible so they can get on with their day. No one wants to wait forever!

\section*{Input}

The input consists of a single test case. The first line contains two space-separated integers $n, m$, where $1 \leq n \leq 10^5$ denotes the number of customers that have lined up to get their cars washed, and $1 \leq m \leq 10^3$ denotes the amount of time (in minutes) Bob has to wash cars before calling it a day.

The next line contains $n$ space-separated integers $a\ (0 \leq a \leq 10^5)$, where $a_i$ denotes the amount of money Bob will receive for washing the $i^\text{th}$ car.

The following line contains $n$ space-separated integers $t\ (1 \leq t \leq 10^3)$, where $t_i$ denotes the amount of time (in minutes) Bob needs to wash the $i^\text{th}$ car.

Finally, the last line contains $n$ space-separated integers $p\ (1 \leq p \leq 10^3)$, where $p_i$ denotes the amount of time (in minutes) the $i^\text{th}$ customer is willing to wait for Bob to \textbf{start} washing their car before leaving.

\section*{Output}

Given Bob's time constraints, output the maximum amount of money he can earn today. 
